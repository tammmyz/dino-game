\documentclass{article}
\usepackage[utf8]{inputenc}
\usepackage{graphics}
\usepackage{amsmath}
\usepackage{graphicx}
\usepackage{geometry}
\usepackage{caption}
\usepackage{url}
\usepackage{booktabs}
\usepackage{tabularx}


\title{Software Requirements Specification: Chrome Dino Runner \\ \bigskip \large SFWRENG 3XA3 Project \\ \bigskip \large Team Number: L03 Group 1 \\ \large Team Name: ``Team Rex'' }

\author{Chelsea Maramot \\ maramotc \\ \\ Anjola Adewale \\ adewaa1 \\ \\ Sheridan Fong \\ fongs7 }

\date{February 11 2022}

\begin{document}

\maketitle
% section 1: Project Drivers ---------------------------
\section{Project Drivers}
\subsection{The Purpose of the Project}
The current pandemic has abruptly disrupted the entertainment sector, and now people are seeking entertainment within the comfort of their homes. An example of an at-home entertainment solution is video games. We aim to provide home entertainment by redesigning the classic Chrome T-Rex dinosaur game. We plan to improve its user-friendliness and interactivity while maintaining the game’s basic functionality. The game's development will follow the software development process and be implemented in python using the PyGame library. The entire process will be documented and tested using the unit test framework. 
\subsection{The Stakeholders}
\subsubsection{The Client}
The clients for this project are the course instructor of SFWRENG 3XA3, Dr. Ashgar Bokhari, and the teaching assistants (TAs), Stephanie Koehl, Veersah Palanichamy and Abdul Rab Mohammed. The clients will provide project requirements, deliverables and deadlines. They will also provide guidance when necessary and evaluate the project with respect to the requirements in the SRS document. 
\subsubsection{The Customers}
The customers for this project are individuals who are interested in playing Chrome-Dino. The project does not explicitly target a demographic but is rather designed as a general-purpose entertainment source. The project is designed for anyone with the game's required software, such as Python and the PyGame library. 
\subsubsection{Other Stakeholders}
All members of group 1 are stakeholders of this project. We are all responsible for the development process, such as implementing, testing, and documenting the project. Group 1 members all care for the project's success and are responsible for maintaining the repository. Developers that fork the repository and continue the project's development are also stakeholders as they will be continuing the project and are interested in the project's success


% Section 2: Project Constraints 
\section{Project Constraints}
are restrictions on the product due to the budget or the 
time available to build the product
\subsection{Mandated Constraints}
\subsubsection{Solution Design Constraints}
Description: The game (an executable file) must be able to run on any machine running on Windows 7 or newer, macOS 10.12 or newer or Linux Ubuntu or newer
Rationale: Most computer users already use systems with these specification and so the users will not need to purchase a new system.
Fit criterion: The game will be developed into an executable file that will be made to run on Windows or newer, macOS Sierra 10.12 or newer or Linux Ubuntu or newer
\subsubsection{Implementation Environment of the Current System}
N/A
\subsubsection{Partner of a Collaborative Application}
N/A
\subsubsection{Off-the-Shelf Software}
N/A
\subsubsection{Anticipated Workable Environment}
N/A
\subsubsection{Schedule Constraints}
Description: This project must follow the project schedule outlined in the Gantt chart and the Task Section.
Rationale: This is because, the project must be completed prior to the end of the course a strict pre-defined plan must be followed. This is also so to ensure that all deliverable are submitted by their due dates
Fit Criterion: All deliverable contained within this project must completed and submitted by April 12, 2022. 
Due to the time constraints on this project(3 months), this project needs to follow a predefined plan in order to ensure it's completion in an organized manner.
\subsubsection{Budget Constraints}
\subsubsection{Enterprise Constraints}
\subsection{Naming Conventions and Terminology}

\begin{table}
\caption{Table of Naming Conventions and Terminology.}
\begin{tabular}{l c p{.5\textwidth}}
\toprule
Term       && Definition \\
\midrule
Chrome Dino Runner  && for all $xy$ iht if $xy \in R$ and $yx \in R$, then $x=y$. \\
\midrule
High Score          && Highest score achieved by the player \\
\midrule
Pygame              && a cross-platform set of Python modules designed for writing video games. \\
\midrule
Python              && The programming language used in developing Chrome Dino Runner . \\
\midrule
Score               && A numerical value which quantifies the player's performance the game. \\
\midrule
SRS       && Acronym for Software Requirements Specification; A document that describes what the system will do and the expected performance. \\
\midrule
User       && The Individual who will be playing  our game. \\
\midrule
WASD keys           && Four keyboard keys that are used to interact with video games in lieu of Arrow keys or a controller. W and S control forward and backward movement, while A and D are left and right \\
\midrule
Basic English Proficiency && $R$ Knowledge of vocabulary words, ability to speak simple phrases or sentences \\
\bottomrule
\end{tabular}
\end{table}

\subsection{Relevant Facts and Assumptions}
\subsubsection{Facts}
The original repository contains 350 lines of code and is developed in Python using the Pygame library. 
\subsubsection{Assumptions}
Users possess a computer running Windows 7 or newer, macOS 10.12 or newer or Linux Ubuntu or newer
User have basic English proficiency 
Users know how to operate a PC
Users have the visual and physical capabilities to play the game



% Section 3: Functional Requirements
\section{Functional Requirements}

\subsection{The Scope of the Work and the Product}



\subsubsection{The Context of the Work}
\subsubsection{The Context of the Work}
% come back to and insert picture properly 
\begin{figure}[!ht]
	\centering
	\includegraphics{context_diagram.png}
	\caption{Caption}
\end{figure}
\subsubsection{Work Partitioning}


\noindent\setlength\tabcolsep{4pt}%
\captionof{table}{Work Partitioning Event Details}
\begin{tabularx}{\linewidth}{|l|c|*{4}{>{\RaggedRight\arraybackslash}X|}}
	\hline
	ID & Event Name & Event Description           & Input                & Expected Output               \\ [0.5ex]
	\hline
	1  & Viewing the Instructions Page  & Displays the instruction page.  & Keyboard/Mouse  & Instructions page \\
	\hline
	2  & Viewing the Leaderboard  & Displays the leaderboard  & Mouse & Leaderboard page  \\
	\hline
	3  & Change Game Settings  & The user can select different themes for the game, changes will be reflected in the UI. They can also alter the audio output. & Theme name and sound settings & Interface display and audio \\
	\hline
	4  & Playing the Game &The user plays the dino game  & Keyboard/Mouse & Interface display, theme music and final display \\
	\hline
\end{tabularx}

\vskip1cm
\subsubsection{Individual Product Use Cases}
\begin{figure}[!ht]
	\centering
	\includegraphics{use_case.png}
	\caption{A use case diagram that displays the functionality of the application.}
\end{figure}

The use case diagram above shows the different ways a user can interact with the system. The user can start a game and the game must include displaying the score and allowing the user to jump over obstacle and displaying the final score. Game play is not impacted on whether a user goes to the main menu or if the score is saved to the database. Therefore, these action extend the use case. The use cases for viewing the instructions page, leaderboard page and settings menu are all self-explanatory. 

\subsection{Functional Requirements}



\textbf{E1: Viewing the Instructions Page} \\
FR. The system must display instructions on how to play the game to the user. \\
FR. The system must provide the user with a way to navigate the main menu. \\

\textbf{E2: Viewing the Leaderboard} \\
FR. The system must display the user with a list of the players with top scores and the respective score.  \\
FR. The system must provide the user with a way to navigate back to the main menu. \\
FR. The system must be able to filter the high scores by player. \\

\textbf{E3: Change Game Settings} \\
FR. The system must provide the user with a way to exit from the game settings to the main menu. \\

FR. The system must allow the user to change the audio settings on the game. Such as having no audio or audio. \\

FR. The system must display the available themes and allow the user to select a theme. \\

\textbf{E4: Playing the Game} \\
FR. The system must display the game to the screen with the appropriate theme. \\

FR. The system must initialize the score to 0 when starting the game. \\

FR. The system must output sound based on the audio settings. \\

FR. The system must start gameplay when the user inputs a keyboard command. \\

FR. The system must allow keyboard inputs to control the user interface. \\

FR. The system must award points to the user after each dino step. \\

FR. The system must display the user’s current score to the screen. \\

FR. The system must allow the user to replay the game. \\

FR. The system must allow the user to stop the game and return to the main menu. \\

FR. The system must allow the user to enter their username when they are done playing the game. \\

FR. After the game is finished, the user score must be entered into the database. \\




% section 4 -----------------------
\section{Non-Functional Requirements}
    \subsection{Look and Feel Requirements}
        \subsubsection{Appearance Requirements}
        LFX. The user interface consist of essential components relevant to the game.
        \subsubsection{Style Requirements} 
    	LFX. The product shall maintain the 90's arcade appearance.\\
    	LFX. The product shall be designed according to the extra themes developed.
   
\subsection{Usability and Humanity Requirements}
    \subsubsection{Ease-Of-Use Requirements}
    UHX. Game can be controlled using two methods, arrow keys and AWSD keys.\\
    UHX. The game must have a simple menu page where game settings can be accessed.\\
    UHX. The game can be used by people intuitively, no training needed and at a maximum basic English level.
    
    \subsubsection{Personalization and Internationalization Requirements}
    PIx. The product shall only be used in English.
    PIX. The user can adjust the theme of the game based on their preferences.
    
    \subsubsection{Learning Requirements}
    LRx. The game will be used without receiving training before using it.\\
    LRx. The game must contain basic instructions within the main menu page.

    \subsubsection{Understandability and Politeness Requirements}
    UPx. The game shall encompass a level of abstraction from the user.\\
    UPx. The game shall use common control keys to play the game.\\
    UPx. The game will include universal symbols and words that are naturally understood by the user community.
    
    \subsubsection{Accessibility Requirements}
    ARx. The game shall be playable for those with colour blindness.
    
\subsection{Performance Requirements}
    \subsubsection{Speed and Latency Requirements}
    PEX. Scores must be uploaded to the leader board in less than 10 seconds.\\
    PEx. The interface must have a maximum response time of 2 seconds.
    PEx. The game shall update the new status parameters within 5 minutes of user input.

    \subsubsection{Safety-Critical Requirements}
    N/A
    
    \subsubsection{Precision or Accuracy Requirements}
    PEx. Integer whole number scores must be uploaded appropriately and displayed on the screen\\
    PEx. Leader board shall be updated according to the top integer score
    
    \subsubsection{Reliability and Availability Requirements}
    N/A

    \subsubsection{Robustness or Fault-Tolerance Requirements}
    N/A
    
    \subsubsection{Capacity Requirements}
    PEx. The game shall only be available to a single player\\
    PEx. The game shall allow a maximum of 3 users stored in the leader board.
    
    \subsubsection{Scalability or Extensibility Requirements}
    PEx. Developers shall be able to add new features, such as themes, without compromising basic functionalities of the game. 
    
    \subsubsection{Longevity Requirements}
    PEx. The game must maintain functionality with existing software until Spring 2022.


\subsection{Operational and Environmental Requirements}
    \subsubsection{Expected Physical Environment}
    PEx. The game can be played with full functionality without internet connection.\\
    PEx. The game can be played within any computer operating system (i.e. Linux, Windows)
    
\subsection{Requirements for Interfacing with Adjacent Systems}
    \subsubsection{Productization Requirements}
    PEX. The game shall be distributed to user computers as a .EXE file.
    PEX. The game shall be installed without the use of separately printed instructions.
    
 \subsection{Release Requirements}
    PEX. The game will be released by April 5, 2021.
    
 \subsection{Maintainability and Support Requirements}
    \subsubsection{Maintenance Requirements}
     MAx. Source code must be up to date on the latest changes.\\
    MAx. Source code must be appropriately documented.
    
    \subsubsection{Supportability Requirements}
    MAX. The full project source code shall be made available to Git users in order to raise issues.
    
    \subsubsection{Adaptability Requirements}
    MAX. The product shall run under Windows 10, Linux Ubuntu 16.04, or newer versions of these operating systems.
    
\subsection{Security Requirements}
    \subsubsection{Access Requirements}
    SRx. The game users shall have read-only access to the leader board and user scores.
    SRx. Users are permitted to change game settings through user interface.
    
    \subsubsection{Integrity Requirements}
    SRx. The product shall protect itself from unauthorized user modifications.
    SRx. The user shall not be allowed to modify scores.
    
    \subsubsection{Privacy Requirements}
    SRx. Each user shall only be able to view other user names and their corresponding high scores within the leader board.
    
    \subsubsection{Audit Requirements}
    N/A
    
    \subsubsection{Immunity Requirements}
    SRX. The product must not be vulnerable to unauthorized or undesirable software programs.
    
\subsection{Cultural and Political Requirements}
    \subsubsection{Cultural Requirements}
    CPx. The game themes shall not be offensive to any cultures or religions. \\
    CPx. The game shall allow users to input non-offensive names only in English.

    \subsubsection{Political Requirements}    N/A
    
\subsection{Legal Requirements}
    \subsubsection{Compliance Requirements}
    N/A
    
    \subsubsection{Standards Requirements} 
    N/A
\subsection{Health and Safety Requirements}
    N/A
%section 5
\section{Project Issues}
\subsection{Open Issues}
\subsection{Off-the-Shelf Solutions}
\subsection{New Problems}
\subsection{Tasks}
\subsection{Migration to the New Product}
\subsection{Risks}
\subsection{Costs}
\subsection{User Documentation and Training}
\subsubsection{Documentation}
\subsubsection{Training}
\subsection{Waiting Room}
\subsection{Ideas for Solutions}
\end{document}
