\documentclass[12pt, titlepage]{article}

\usepackage{fullpage}
\usepackage[round]{natbib}
\usepackage{multirow}
\usepackage{booktabs}
\usepackage{tabularx}
\usepackage{graphicx}
\usepackage{float}
\usepackage{hyperref}
\hypersetup{
    colorlinks,
    citecolor=black,
    filecolor=black,
    linkcolor=red,
    urlcolor=blue
}
\usepackage[round]{natbib}

\newcounter{acnum}
\newcommand{\actheacnum}{AC\theacnum}
\newcommand{\acref}[1]{AC\ref{#1}}

\newcounter{ucnum}
\newcommand{\uctheucnum}{UC\theucnum}
\newcommand{\uref}[1]{UC\ref{#1}}

\newcounter{mnum}
\newcommand{\mthemnum}{M\themnum}
\newcommand{\mref}[1]{M\ref{#1}}

\title{SE 3XA3: Software Requirements Specification\\Title of Project}

\author{Team \#1, Team Name
		\\ Anjola Adewale and adewaa1
		\\ Sheridan Fong and fongs7
		\\ Chelsea Maramot and maramotc
}

\date{\today}

% \input{../../Comments}

\begin{document}

\maketitle

\pagenumbering{roman}
\tableofcontents
\listoftables
\listoffigures

\begin{table}[bp]
\caption{\bf Revision History}
\begin{tabularx}{\textwidth}{p{3cm}p{2cm}X}
\toprule {\bf Date} & {\bf Version} & {\bf Notes}\\
\midrule
Date 1 & 1.0 & Notes\\
Date 2 & 1.1 & Notes\\
\bottomrule
\end{tabularx}
\end{table}

\newpage

\pagenumbering{arabic}


\section{Introduction}

Decomposing a system into modules is a commonly accepted approach to developing
software.  A module is a work assignment for a programmer or programming
team~\citep{ParnasEtAl1984}.  We advocate a decomposition
based on the principle of information hiding~\citep{Parnas1972a}.  This
principle supports design for change, because the ``secrets'' that each module
hides represent likely future changes.  Design for change is valuable in SC,
where modifications are frequent, especially during initial development as the
solution space is explored.  

Our design follows the rules layed out by \citet{ParnasEtAl1984}, as follows:
\begin{itemize}
\item System details that are likely to change independently should be the
  secrets of separate modules.
\item Each data structure is used in only one module.
\item Any other program that requires information stored in a module's data
  structures must obtain it by calling access programs belonging to that module.
\end{itemize}

After completing the first stage of the design, the Software Requirements
Specification (SRS), the Module Guide (MG) is developed~\citep{ParnasEtAl1984}. The MG
specifies the modular structure of the system and is intended to allow both
designers and maintainers to easily identify the parts of the software.  The
potential readers of this document are as follows:

\begin{itemize}
\item New project members: This document can be a guide for a new project member
  to easily understand the overall structure and quickly find the
  relevant modules they are searching for.
\item Maintainers: The hierarchical structure of the module guide improves the
  maintainers' understanding when they need to make changes to the system. It is
  important for a maintainer to update the relevant sections of the document
  after changes have been made.
\item Designers: Once the module guide has been written, it can be used to
  check for consistency, feasibility and flexibility. Designers can verify the
  system in various ways, such as consistency among modules, feasibility of the
  decomposition, and flexibility of the design.
\end{itemize}

The rest of the document is organized as follows. Section
\ref{SecChange} lists the anticipated and unlikely changes of the software
requirements. Section \ref{SecMH} summarizes the module decomposition that
was constructed according to the likely changes. Section \ref{SecConnection}
specifies the connections between the software requirements and the
modules. Section \ref{SecMD} gives a detailed description of the
modules. Section \ref{SecTM} includes two traceability matrices. One checks
the completeness of the design against the requirements provided in the SRS. The
other shows the relation between anticipated changes and the modules. Section
\ref{SecUse} describes the use relation between modules.

\section{Anticipated and Unlikely Changes} \label{SecChange}

This section lists possible changes to the system. According to the likeliness
of the change, the possible changes are classified into two
categories. Anticipated changes are listed in Section \ref{SecAchange}, and
unlikely changes are listed in Section \ref{SecUchange}.

\subsection{Anticipated Changes} \label{SecAchange}

Anticipated changes are the source of the information that is to be hidden
inside the modules. Ideally, changing one of the anticipated changes will only
require changing the one module that hides the associated decision. The approach
adapted here is called design for
change.

\begin{description}
\item[\refstepcounter{acnum} \actheacnum \label{acHardware}:] The specific
  hardware on which the software is running.
\item[\refstepcounter{acnum} \actheacnum \label{acInput}:] The format of the
  initial input data.
\item[\refstepcounter{acnum} \actheacnum \label{acInput}:] The grammar of python through future releases or the inclusion of older python grammars
\item[\refstepcounter{acnum} \actheacnum \label{acInput}:] New funtionality of the Pygame library through future releases 
\item[\refstepcounter{acnum} \actheacnum \label{acInput}:] The format of the final packaging and distribution format of the program
\item[\refstepcounter{acnum} \actheacnum \label{acInput}:] The graphics are likely to change for obstacle and character based on themes and graphical design decisions. 
\end{description}

\subsection{Unlikely Changes} \label{SecUchange}

The module design should be as general as possible. However, a general system is
more complex. Sometimes this complexity is not necessary. Fixing some design
decisions at the system architecture stage can simplify the software design. If
these decision should later need to be changed, then many parts of the design
will potentially need to be modified. Hence, it is not intended that these
decisions will be changed.

\begin{description}
\item[\refstepcounter{ucnum} \uctheucnum \label{ucIO}:] Input/Output devices
  (Input: File and/or Keyboard, Output: File, Memory, and/or Screen).
\item[\refstepcounter{ucnum} \uctheucnum \label{ucInput}:] There will always be
  a source of input data external to the software.
\item[\refstepcounter{ucnum} \uctheucnum \label{ucInput}:] The algorithms for the game points calculation and leaderboard display
\item[\refstepcounter{ucnum} \uctheucnum \label{ucInput}:] The objective of the game which is to score as many points as possible. 
\item ...
\end{description}


\section{Module Hierarchy} \label{SecMH}

This section provides an overview of the module design. Modules are summarized
in a hierarchy decomposed by secrets in Table \ref{TblMH}. The modules listed
below, which are leaves in the hierarchy tree, are the modules that will
actually be implemented.

\begin{description}
\item [\refstepcounter{mnum} \mthemnum \label{mHH}:] Obstacle Module
\item [\refstepcounter{mnum} \mthemnum \label{mHH}:] Character Module
\item [\refstepcounter{mnum} \mthemnum \label{mHH}:] Cloud Module
\item [\refstepcounter{mnum} \mthemnum \label{mHH}:] Global Variable Module
\item [\refstepcounter{mnum} \mthemnum \label{mHH}:] Large Obstacle Module
\item [\refstepcounter{mnum} \mthemnum \label{mHH}:] Small Obstacle Module
\item [\refstepcounter{mnum} \mthemnum \label{mHH}:] Images Module
\item [\refstepcounter{mnum} \mthemnum \label{mHH}:] Bird Module
\item [\refstepcounter{mnum} \mthemnum \label{mHH}:] Instructions display Module
\item [\refstepcounter{mnum} \mthemnum \label{mHH}:] ChromeDino Module
\item [\refstepcounter{mnum} \mthemnum \label{mHH}:] Leaderboard Module
\item [\refstepcounter{mnum} \mthemnum \label{mHH}:] Leaderboard display Module
\item [\refstepcounter{mnum} \mthemnum \label{mHH}:] Game Settings Display

\end{description}


% Behaviour hiding may hide input formats, screen formats, and text messages
% Software decision hiding may hide internal data structures and algorithms


\begin{table}[h!]
    \centering
    \begin{tabular}{p{0.3\textwidth} p{0.6\textwidth}}
    \toprule
    \textbf{Level 1} & \textbf{Level 2}\\
    \midrule
    
    {Hardware-Hiding Module} & Python, Visual Studio\\
    \midrule
    
    \multirow{7}{0.3\textwidth}{Behaviour-Hiding Module} 
    & Global Variable Module \\
    & Images Module \\
    & Instructions Display Module \\
    & Leaderboard Display Module \\
    & Game Settings Display\\ 
    \midrule
    
    \multirow{3}{0.3\textwidth}{Software Decision Module} & ChromeDino Module\\
    & Obstacle Module \\
    & Small Obstacle Module \\
    & Large Obstacle Module \\
    & Character Module \\
    & Cloud Module \\
    & Bird Module \\
    & Leaderboard Module \\
    \bottomrule
    
    \end{tabular}
    \caption{Module Hierarchy}
    \label{TblMH}
    \end{table}
    
    \begin{table}[h!]
    \centering
    \begin{tabular}{p{0.3\textwidth} p{0.6\textwidth}}
    \toprule
    \textbf{Level 1} & \textbf{Level 2}\\
    \midrule
    
    
    \multirow{7}{0.3\textwidth}{Model} 
    & Obstacle Module \\
    & Small Obstacle Module \\
    & Large Obstacle Module \\
    & Character Module \\
    & Cloud Module \\
    & Bird Module \\
    & Leaderboard Module \\
    \midrule
    
    \multirow{3}{0.3\textwidth}{View} 
    & Global Variable Module \\
    & Images Module \\
    & Instructions Display Module \\
    & Leaderboard Display Module \\
    & Game Settings Display\\ 
    \bottomrule
    
    \multirow{1.2}{0.3\textwidth}{Controller} 
    & ChromeDino Module \\
    \midrule
    
    \end{tabular}
    \caption{Module Hierarchy: MVC Model}
    \label{TblMH}
    \end{table}
    
    \section{Connection Between Requirements and Design} \label{SecConnection}
    
    The design of the system is intended to satisfy the requirements developed in
    the SRS. In this stage, the system is decomposed into modules. The connection
    between requirements and modules is listed in Table \ref{TblRT}.

    
    