\documentclass[12pt, titlepage]{article}

\usepackage{booktabs}
\usepackage{tabularx}
\usepackage{hyperref}
\hypersetup{
    colorlinks,
    citecolor=black,
    filecolor=black,
    linkcolor=red,
    urlcolor=blue
}
\usepackage[round]{natbib}

\title{SE 3XA3: Test Plan\\Title of Project}

\author{Team \#, Team Name
		\\ Student 1 name and macid
		\\ Student 2 name and macid
		\\ Student 3 name and macid
}

\date{\today}

\input{../Comments}

\begin{document}

\maketitle

\pagenumbering{roman}
\tableofcontents
\listoftables
\listoffigures

\begin{table}[bp]
\caption{\bf Revision History}
\begin{tabularx}{\textwidth}{p{3cm}p{2cm}X}
\toprule {\bf Date} & {\bf Version} & {\bf Notes}\\
\midrule
Date 1 & 1.0 & Notes\\
Date 2 & 1.1 & Notes\\
\bottomrule
\end{tabularx}
\end{table}

\newpage

\pagenumbering{arabic}

This document ...

\section{General Information}

\subsection{Purpose}

\subsection{Scope}

\subsection{Acronyms, Abbreviations, and Symbols}
	
\begin{table}[hbp]
\caption{\textbf{Table of Abbreviations}} \label{Table}

\begin{tabularx}{\textwidth}{p{3cm}X}
\toprule
\textbf{Abbreviation} & \textbf{Definition} \\
\midrule
Abbreviation1 & Definition1\\
Abbreviation2 & Definition2\\
\bottomrule
\end{tabularx}

\end{table}

\begin{table}[!htbp]
\caption{\textbf{Table of Definitions}} \label{Table}

\begin{tabularx}{\textwidth}{p{3cm}X}
\toprule
\textbf{Term} & \textbf{Definition}\\
\midrule
Term1 & Definition1\\
Term2 & Definition2\\
\bottomrule
\end{tabularx}

\end{table}	

\subsection{Overview of Document}

\section{Plan}
	
\subsection{Software Description}

\subsection{Test Team}

\subsection{Automated Testing Approach}

\subsection{Testing Tools}

\subsection{Testing Schedule}
		
See Gantt Chart at the following url ...

\section{System Test Description}
	
\subsection{Tests for Functional Requirements}

\subsubsection{Area of Testing1}
		
\paragraph{Title for Test}

\begin{enumerate}

\item{test-id1\\}

Type: Functional, Dynamic, Manual, Static etc.
					
Initial State: 
					
Input: 
					
Output: 
					
How test will be performed: 
					
\item{test-id2\\}

Type: Functional, Dynamic, Manual, Static etc.
					
Initial State: 
					
Input: 
					
Output: 
					
How test will be performed: 

\end{enumerate}

\subsubsection{Area of Testing2}



\subsubsection{Instructions Page Tests}

\begin{enumerate}

\item{test-id1\\}

Type: Functional, Dynamic, Unit and Manual test.
					
Initial State: Current page displayed within interface is the main menu page and instructions = False.
					
Input: User selects the area within the "How to play" text
					
Output: The current page displayed on the interface will be the instructions page and instructions = True.
					
How test will be performed: The program will be given a specific user input. The response will be compared to the expected output: user is taken to the settings page. For unit testing, the state of the instructions variable will be asserted.


\item{test-id2\\}

Type: Functional, Dynamic, and Manual test.
					
Initial State: Current page displayed within the interface is the instructions page.
					
Input: The user enters the associated keyboard exit key("e").
					
Output: Current page displayed within the interface will transition to the main menu page.
					
How test will be performed: The user will enter the appropriate input. The response will be compared to the expected output: user is taken back to the main menu page.


\item{test-id2\\}

Type: Functional, Dynamic, and Manual test.
					
Initial State:  The game is paused and instructions = False.
					
Input: The user enters the associated keyboard("i") input.
					
Output: Current page displayed within the interface will transition to the instructions page and instructions = True.
					
How test will be performed: The user will enter the appropriate input. The response will be compared to the expected output: user is taken back to the instructions page. For unit testing, the state of the instructions variable will be asserted.


\item{test-id2\\}

Type: Functional, Dynamic, Unit and Manual test.
					
Initial State:  The current clock time is between 7:00 p.m. and 7:00 a.m and the instructions page is currently displayed on the interface.
					
Input: N/A
					
Output: Current page displayed within the interface will be the instructions page in dark mode.
					
How test will be performed: The displayed instructions page will be visually inspected to check if it is in dark mode. Similarly, the state of the background colour and font colours will be asserted to their expected values in unit testing. 



\item{test-id2\\}

Type: Automated, manual, dynamic, and functional test.
					
Initial State: Current page displayed within the interface is the instructions page.
					
Input: N/A
					
Output: Text elements of the instructions page are displayed.
					
How test will be performed: A manual test will be performed to inspect if the text elements are in their correct positions. Unit testing will assert the element positions with their expected positions. 

\end{enumerate}



\subsection{Tests for Nonfunctional Requirements}

\subsubsection{Area of Testing1}
		
\paragraph{Title for Test}

\begin{enumerate}

\item{test-id1\\}

Type: 
					
Initial State: 
					
Input/Condition: 
					
Output/Result: 
					
How test will be performed: 
					
\item{test-id2\\}

Type: Functional, Dynamic, Manual, Static etc.
					
Initial State: 
					
Input: 
					
Output: 
					
How test will be performed: 

\end{enumerate}

\subsubsection{Area of Testing2}

...

\subsection{Traceability Between Test Cases and Requirements}

\section{Tests for Proof of Concept}

\subsection{Area of Testing1}
		
\paragraph{Title for Test}

\begin{enumerate}

\item{test-id1\\}

Type: Functional, Dynamic, Manual, Static etc.
					
Initial State: 
					
Input: 
					
Output: 
					
How test will be performed: 
					
\item{test-id2\\}

Type: Functional, Dynamic, Manual, Static etc.
					
Initial State: 
					
Input: 
					
Output: 
					
How test will be performed: 

\end{enumerate}

\subsection{Area of Testing2}

...

	
\section{Comparison to Existing Implementation}	
				
\section{Unit Testing Plan}


The unittest unit testing framework will be used to conduct unit tests.
		
\subsection{Unit testing of internal functions}

	
To validate internal functions, test cases will be created for each class, as well as specific functions that return a value. Test cases will cover edge cases to validate the extreme boundaries of the various input. Additionally, exceptions and basic tests will be inputs to testing internal functions. The tests will examine the correctness and robustness of the program.  Furthermore, no stubs or drivers are needed in the execution of test cases. The classes should import the necessary libraries and modules needed for testing. We will ensure that majority {i.e. at least 80\%} of the functions will be covered by the test cases through coverage metrics. 


\subsection{Unit testing of output files}		

The only output file generated by the game is the score.txt file which stores the high scores displayed on the leader board. To validate the file output, a combination of unit testing and manual testing will be done. A dynamic manual test is conducted by having the user play the game and beat the high score. Their current score should be displayed in the leaderboard and found on top of the score.txt file. On the other hand, unit testing will verify that the high score displayed on the interface is the same as the high score in the score.txt file. 

Asides from the output file, the game has a graphical user interface generated through PyGame. A user conducted black box testing will be done to test the interface and its functionalities. The user will play the game, testing each feature. If the game is successfully played by the user and all functionalities stated within the SRS have been achieved, then the user interface passes the test.


\bibliographystyle{plainnat}

\bibliography{SRS}

\newpage

\section{Appendix}

This is where you can place additional information.

\subsection{Symbolic Parameters}

The definition of the test cases will call for SYMBOLIC\_CONSTANTS.
Their values are defined in this section for easy maintenance.

\subsection{Usability Survey Questions?}

This is a section that would be appropriate for some teams.

\end{document}