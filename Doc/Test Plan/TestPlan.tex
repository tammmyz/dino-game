\documentclass[12pt, titlepage]{article}

\usepackage{booktabs}
\usepackage{tabularx}
\usepackage{hyperref}
\hypersetup{
    colorlinks,
    citecolor=black,
    filecolor=black,
    linkcolor=red,
    urlcolor=blue
}
\usepackage[round]{natbib}

\title{SE 3XA3: Test Plan\\Title of Project}

\author{Team \#, Team Name
		\\ Student 1 name and macid
		\\ Student 2 name and macid
		\\ Student 3 name and macid
}

\date{\today}

\input{../Comments}

\begin{document}

\maketitle

\pagenumbering{roman}
\tableofcontents
\listoftables
\listoffigures

\begin{table}[bp]
\caption{\bf Revision History}
\begin{tabularx}{\textwidth}{p{3cm}p{2cm}X}
\toprule {\bf Date} & {\bf Version} & {\bf Notes}\\
\midrule
Date 1 & 1.0 & Notes\\
Date 2 & 1.1 & Notes\\
\bottomrule
\end{tabularx}
\end{table}

\newpage

\pagenumbering{arabic}

This document ...

\section{General Information}

\subsection{Purpose}

\subsection{Scope}

\subsection{Acronyms, Abbreviations, and Symbols}
	
\begin{table}[hbp]
\caption{\textbf{Table of Abbreviations}} \label{Table}

\begin{tabularx}{\textwidth}{p{3cm}X}
\toprule
\textbf{Abbreviation} & \textbf{Definition} \\
\midrule
Abbreviation1 & Definition1\\
Abbreviation2 & Definition2\\
\bottomrule
\end{tabularx}

\end{table}

\begin{table}[!htbp]
\caption{\textbf{Table of Definitions}} \label{Table}

\begin{tabularx}{\textwidth}{p{3cm}X}
\toprule
\textbf{Term} & \textbf{Definition}\\
\midrule
Term1 & Definition1\\
Term2 & Definition2\\
\bottomrule
\end{tabularx}

\end{table}	

\subsection{Overview of Document}

\section{Plan}
	
\subsection{Software Description}

\subsection{Test Team}

The test team consists of Anjola Adewale, Sheridan Fong, and Chelsea Maramot. The testers will divide the tests according to the features of CDR and cover all the necessary types of testing.

\subsection{Automated Testing Approach}

The software required is a Python automated testing framework and a text editor to modify code. The personnel responsible for testing setup should have sufficient knowledge of the Python testing framework in order to write a series of test cases. In contrast, for manual testing, the user will not need previous knowledge of Python and the implementation of the game. The Python automated testing framework will be unittest which provides its own documentation found online. Unittest provides tools creating and running test case. Each test case will begin with "test". Each test case will implement an assert statement to check expected results and raise exceptions. The results of these tests will be displayed through the terminal.


\subsection{Testing Tools}

Asides from the output file, the game has a graphical user interface generated through PyGame. A user conducted black box testing will be done to test the interface and its functionalities. The user will play the game, testing each feature. If the game is successfully played by the user and all functionalities stated within the SRS have been achieved, then the user interface passes the test.


\subsection{Testing Schedule}
		
See Gantt Chart at the following url ...

\section{System Test Description}
	
\subsection{Tests for Functional Requirements}

\subsubsection{Area of Testing1}
		
\paragraph{Title for Test}

\begin{enumerate}

\item{test-id1\\}

Type: Functional, Dynamic, Manual, Static etc.
					
Initial State: 
					
Input: 
					
Output: 
					
How test will be performed: 
					
\item{test-id2\\}

Type: Functional, Dynamic, Manual, Static etc.
					
Initial State: 
					
Input: 
					
Output: 
					
How test will be performed: 

\end{enumerate}

\subsubsection{Leaderboard Tests}

\begin{enumerate}

\item{test-id1\\}

Type: Functional, Dynamic, Manual
					
Initial State: User is at the Restart screen waiting for an area to be selected(mouse event)
					
Input: User clicks on an area within the area of the "Leaderboard" text
					
Output: The Leaderboard page is displayed and contains up to five users and their scores.
					
How test will be performed: After the menu function is called and the restart\_flag variable is set to true, the appropriate user input(mouse event) is entered and the output is checked to ensure that the leaderboard is displayed. 

\item{test-id2\\}

Type: Automated, Dynamic, Structural
					
Initial State: None, this function is called on a unit basis. However the score.txt file must be populated
					
Input: Leaders\_text (Python list containing top users and their score)
					
Output: Unit test passes if the users in the leader board have the highest recorded scores
					
How test will be performed: The score.txt file will be manually populated with users and their scores.
After calling the get\_leaders() function, each index of the leader\_txt output is validated using assert statements.

\item{test-id2\\}

Type: Functional, Manual, Dynamic 
					
Initial State: display\_leaderboard flag is set to true and the Leader board page is displayed
					
Input: N/A
					
Output: Text elements of the Leaderboard page are properly displayed
					
How test will be performed: When the Leaderboard page is displayed, manually inspect the page to 
ensure that there is no text area overlap between the individual texts containing the user name and score. 

\end{enumerate}

\subsubsection{Instructions Page Tests}

\begin{enumerate}

\item{test-id1\\}

Type: Functional, Dynamic, Unit and Manual test.
					
Initial State: Current page displayed within interface is the main menu page and instructions = False.
					
Input: User selects the area within the "How to play" text
					
Output: The current page displayed on the interface will be the instructions page and instructions = True.
					
How test will be performed: The program will be given a specific user input. The response will be compared to the expected output: user is taken to the settings page. For unit testing, the state of the instructions variable will be asserted.


\item{test-id2\\}

Type: Functional, Dynamic, and Manual test.
					
Initial State: Current page displayed within the interface is the instructions page.
					
Input: The user enters the associated keyboard exit key("e").
					
Output: Current page displayed within the interface will transition to the main menu page.
					
How test will be performed: The user will enter the appropriate input. The response will be compared to the expected output: user is taken back to the main menu page.


\item{test-id2\\}

Type: Functional, Dynamic, and Manual test.
					
Initial State:  The game is paused and instructions = False.
					
Input: The user enters the associated keyboard("i") input.
					
Output: Current page displayed within the interface will transition to the instructions page and instructions = True.
					
How test will be performed: The user will enter the appropriate input. The response will be compared to the expected output: user is taken back to the instructions page. For unit testing, the state of the instructions variable will be asserted.


\item{test-id2\\}

Type: Functional, Dynamic, Unit and Manual test.
					
Initial State:  The current clock time is between 7:00 p.m. and 7:00 a.m and the instructions page is currently displayed on the interface.
					
Input: N/A
					
Output: Current page displayed within the interface will be the instructions page in dark mode.
					
How test will be performed: The displayed instructions page will be visually inspected to check if it is in dark mode. Similarly, the state of the background colour and font colours will be asserted to their expected values in unit testing. 



\item{test-id2\\}

Type: Automated, manual, dynamic, and functional test.
					
Initial State: Current page displayed within the interface is the instructions page.
					
Input: N/A
					
Output: Text elements of the instructions page are displayed.
					
How test will be performed: A manual test will be performed to inspect if the text elements are in their correct positions. Unit testing will assert the element positions with their expected positions. 

\end{enumerate}



\subsection{Tests for Nonfunctional Requirements}

\subsubsection{Area of Testing1}
		
\paragraph{Title for Test}

\begin{enumerate}

\item{test-id1\\}

Type: 
					
Initial State: 
					
Input/Condition: 
					
Output/Result: 
					
How test will be performed: 
					
\item{test-id2\\}

Type: Functional, Dynamic, Manual, Static etc.
					
Initial State: 
					
Input: 
					
Output: 
					
How test will be performed: 

\end{enumerate}

\subsubsection{Area of Testing2}

...

\subsection{Traceability Between Test Cases and Requirements}

\section{Tests for Proof of Concept}

Proof of Concept verifies and validates the method by which automated testing is performed. The testing for proof of concept will test the functionality of the game and game flow logic. 

\subsection{Game Play}
		
\paragraph{Title for Test}

\begin{enumerate}

\item{test-id1\\}

Type: Functional, Dynamic, Manual, Static etc.
					
Initial State: 
					
Input: 
					
Output: 
					
How test will be performed: 
					
\item{test-id2\\}

Type: Functional, Dynamic, Manual, Static etc.
					
Initial State: 
					
Input: 
					
Output: 
					
How test will be performed: 

\end{enumerate}

\subsection{Additional Game Pages}



\paragraph{Home Page}

\begin{enumerate}
	
	\item{test-id1\\}
	
	Type: Functional, Dynamic, Manual. 
	
	Initial State: Main Page and default settings (theme and audio)
	
	Input: any keyboard input 
	
	Output: Game track is shown to the screen. 
	
	How test will be performed: Manual and dynamic testing will be used to ensure the program functions as expected. Visual inspection will validate if the screen is displaying the game play page. 
	
	\item{test-id2\\}
	
	Type: Functional, Dynamic, Manual.
	
	Initial State: Main page and default settings (theme and audio) 
	
	Input: Mouse click on "Game Settings" text
	
	Output: Settings page. 
	
	How test will be performed:Manual and dynamic testing will be used to ensure the program functions as expected. Visual inspection will validate if the screen is displaying the settings.  
	
	\item{test-id3\\}
	
	Type: Functional, Dynamic, Manual.
	
	Initial State: Main page and default settings (theme and audio) 
	
	Input: Mouse click on "How to Play" text
	
	Output: Instructions page
	
	How test will be performed: Manual and dynamic testing will be used to ensure the program functions as expected. Visual inspection will validate if the screen is displaying the instructions page. 
	
\end{enumerate}

\paragraph{Instructions Page}

\begin{enumerate}
	
	\item{test-id1\\}
	
	Type: Functional, Dynamic, Manual. 
	
	Initial State: Instructions Page
	
	Input: Keyboard input "1", "2" or "3" 
	
	Output: Graphics on the main page and game track change. 
	
	How test will be performed: Manual and dynamic testing will be used to ensure the program functions as expected. Visual inspection will validate if the graphics have changed according to the theme. On the main page the icon will change and on the game track the obstacles and character will change based on the theme. 
	
	\item{test-id2\\}
	
	Type: Functional, Dynamic, Manual.
	
	Initial State: Instructions Page
	
	Input: Keyboard input "N" or "A" 
	
	Output: if "N" audio = True, if "A" audio = False
	
	How test will be performed: Manual and dynamic testing will be used to ensure the program functions as expected. Manual inspection testing will be used to check if the variable has changed in the appropriate python file. 
	
	\item{test-id3\\}
	
	Type: Functional, Dynamic, Manual.
	
	Initial State: Instructions Page
	
	Input: Keyboard input "E"
	
	Output: Main Page
	
	How test will be performed: Manual and dynamic testing will be used to ensure the program functions as expected. Manual inspection testing will be used to check if the home page is displayed. 
	
\end{enumerate}

	
\section{Comparison to Existing Implementation}	
				
\section{Unit Testing Plan}


The unittest unit testing framework will be used to conduct unit tests.
		
\subsection{Unit testing of internal functions}

	
To validate internal functions, test cases will be created for each class, as well as specific functions that return a value. Test cases will cover edge cases to validate the extreme boundaries of the various input. Additionally, exceptions and basic tests will be inputs to testing internal functions. The tests will examine the correctness and robustness of the program.  Furthermore, no stubs or drivers are needed in the execution of test cases. The classes should import the necessary libraries and modules needed for testing. We will ensure that majority {i.e. at least 80\%} of the functions will be covered by the test cases through coverage metrics. 


\subsection{Unit testing of output files}		

The only output file generated by the game is the score.txt file which stores the high scores displayed on the leader board. To validate the file output, a combination of unit testing and manual testing will be done. A dynamic manual test is conducted by having the user play the game and beat the high score. Their current score should be displayed in the leaderboard and found on top of the score.txt file. On the other hand, unit testing will verify that the high score displayed on the interface is the same as the high score in the score.txt file. 

Asides from the output file, the game has a graphical user interface generated through PyGame. A user conducted black box testing will be done to test the interface and its functionalities. The user will play the game, testing each feature. If the game is successfully played by the user and all functionalities stated within the SRS have been achieved, then the user interface passes the test.


\bibliographystyle{plainnat}

\bibliography{SRS}

\newpage

\section{Appendix}

This is where you can place additional information.

\subsection{Symbolic Parameters}

The definition of the test cases will call for SYMBOLIC\_CONSTANTS.
Their values are defined in this section for easy maintenance.

\subsection{Usability Survey Questions?}

This is a section that would be appropriate for some teams.

\end{document}